\section{Auswertung}
\label{sec:Auswertung}

\subsection{Elastizitätsmodul des eckigen Stabes}
\label{subsec:elastiEckig}

\subsubsection{Einseitige Einspannung}
\label{subsubsec:einsEck}
Es wird der Elastizitätsmodul $E$ eines eckigen Stabes berechnet.
Aus Gleichung (\ref{eqn:Durchbiegung}) folgt ein Zusammenhang zwischen dem Drehmoment $D(x)$ und dem Elastizitätsmodul $E$. Um letzteres zu bestimmen
wird eine Hilfsvariable $\eta(x)$ eingeführt, mit 
\begin{align}
  \eta(x) = Lx^2- \frac 13 x^3.
  \label{eqn:eta1}
\end{align}
Wird $\eta(x)$ in Gleichung (\ref{eqn:Durchbiegung}) eingesetzt, folgt ein linearer Zusammenhang zwischen $D(x)$ und $\eta(x)$,
\begin{align}
  D(\eta) = \frac{F}{2EI}\eta.
  \label{eqn:rel1}
\end{align}
Mithilfe einer linearen Regressionskurve wird ein Wert für die Steigung $\mu$ ermittelt, aus dem der Elastizitätsmodul bestimmt
werden kann. Es wird Formel (\ref{eqn:rel1}) umgestellt zu
\begin{align}
  E = \frac{F}{2I\mu}.
  \label{eqn:linRegq}
\end{align}
In \autoref{fig:plot1} ist der Plot zu den Messwerten 
aus \autoref{tab:werte1} zu entnehmen, wobei das Gewicht für die Messwerte von $D_{\text G}(x)$ eine Masse von 
$m_{\text{Last}}=\SI{750}{\gram}$ hat und an der Position $x = \SI{27.5}{\centi\meter}$ positioniert wurde.  


\sloppy
\begin{table}[H]
  \centering
  \caption{Messung des eckigen Stabes bei einseitger Einspannung ($ m_{\text{Last}} = \SI{750}{\gram}$).}
  \label{tab:werte1}
  \begin{tabular}{c c c}
    \toprule
    x / $\si{\centi\meter} $ & $ D_0(x) / \si{\centi\meter}$ & $D_{\text G}(x) / \si{\milli\meter}$ \\
    \midrule
    3 & 7,79 & 7,76 \\
    6 & 7,79 & 7,70 \\
    9 & 7,78 & 7,57 \\
    12 & 7,71 & 7,43 \\
    15 & 7,64 & 7,20 \\
    18 & 7,58 & 6,97 \\
    21 & 7,58 & 6,76 \\
    24 & 7,46 & 6,46 \\
    27 & 7,36 & 6,13 \\
    30 & 7,27 & 5,82 \\
    33 & 7,19 & 5,48 \\
    36 & 7,13 & 5,12 \\
    39 & 6,95 & 4,71 \\
    42 & 6,80 & 4,31 \\
    45 & 6,63 & 3,85 \\
    48 & 6,49 & 3,49 \\
    \bottomrule
  \end{tabular}
\end{table}

\sloppy
\begin{figure}
  \centering
  \includegraphics[width=0.75\textwidth]{build/plot1.pdf}
  \caption{Messung eines quadratischen Stabes bei einseitger Einspannung ($ m_{\text{Last}} = \SI{750}{\gram}$).}
  \label{fig:plot1}
\end{figure}

\sloppy
\begin{table}[H]
  \centering
  \caption{Abmaße des eckigen Stabes.}
  \label{tab:eckigStab}
  \begin{tabular}{c c c}
    \toprule
    m / g & l / mm & a / mm \\
    \midrule
    535.4 & 10.0 & 602.0 \\
    536.2 & 10.0 & 602.0 \\
    536.3 & 10.0 & 602.0 \\
    535.7 & 10.0 & 602.0 \\
    536.1 & 10.0 & 602.0 \\
    \bottomrule
  \end{tabular}
\end{table}

\sloppy
Aus \autoref{tab:eckigStab} sind die Abmaße des eckigen Stabes zu entnehmen. Das Gewicht des Stabes ergibt sich mit \autoref{tab:eckigStab} zu 
\begin{align*}
  \overline{m_{\text{quadr}}} = \SI{535,7 (0,7)}{\gram},
\end{align*}
woraus die Gewichtskraft zu 
\begin{align*}
  F_{\text{Last 1}} = \SI{7,3575}{\newton}
\end{align*} 
folgt. Aus Formel (\ref{eqn:quadrat}) wird das Flächenträgheitsmoment berechnet zu
\begin{align*}
  I_{\square} = \SI{0.83e-3}{\kilo\gram\meter^2}.
\end{align*}
Die lineare Regression wird mithilfe der Python-Erweiterungen numpy \cite{numpy} und scipy \cite{scipy} durchgeführt, während
der Plot mit matplotlib \cite{matplotlib} erstellt wird. Die Erweiterungen liefern eine Ausgleichsgerade vom Typ $D(x) = \mu\cdot\eta(x)+b$
mit den Parametern
\begin{align*}
  \mu = 0.0297 \pm 0.0004, \\
  b = 0.0001 \pm 0.0004. \\
\end{align*}
Das Elastizitätsmodul berechnet sich mit (\ref{eqn:rel1}) zu
\begin{align*}
  E = \SI{149.2(2.0)e9}{\Pa}.
\end{align*}

\subsubsection{Beidseitige Einspannung}
\label{subsubsec:beidsEck}
Aus \autoref{tab:werte3} sind die Messwerte der Biegung eines eckigen Stabes bei beidseitiger Einspannung unter einem Gewicht von
$m_{\text{Last}} = \SI{1550}{\gram}$, welches an der Position $x= \SI{27.5}{\cm}$ positioniert wird, zu entnehmen.
\autoref{fig:plot2} ist Plot zu den Messwerten aus \autoref{tab:werte3} und die lineare Regression zu entnehmen.



\sloppy
\begin{table}[H]
  \centering
  \caption{Messung eines eckigen Stabes bei beidseitiger Einspannung ($m_{\text{Last}} = \SI{1550}{\gram}$).}
  \label{tab:werte3}
  \begin{tabular}{c c c}
    \toprule
    x / $\si{\centi\meter} $ & $ D_0(x) / \si{\centi\meter}$ & $D_{\text G}(x) / \si{\milli\meter}$ \\
    \midrule
    3 & 8,90 & 8,75 \\
    6 & 8,99 & 8,76 \\
    9 & 9,07 & 8,76 \\
    12 & 9,14 & 8,76 \\
    15 & 9,22 & 8,77 \\
    18 & 9,31 & 8,74 \\
    21 & 9,39 & 8,83 \\
    24 & 9,46 & 8,87 \\
    27 & 9,55 & 8,92 \\
    30 & 7,19 & 7,88 \\
    33 & 7,11 & 7,67 \\
    36 & 8,02 & 7,53 \\
    39 & 8,13 & 7,32 \\
    42 & 8,11 & 7,05 \\
    45 & 8,38 & 8,05 \\
    48 & 8,45 & 8,17 \\
    51 & 8,58 & 8,45 \\
    54 & 8,73 & 8,68 \\
    \bottomrule
  \end{tabular}
\end{table}

\sloppy
\begin{figure}[H]
  \centering
  \includegraphics[width=0.75\textwidth]{build/plot2.pdf}
  \caption{Messung eines quadratischen Stabes bei beidseiger Einspannung ($m_{\text{Last}} = \SI{1550}{\gram}$).}
  \label{fig:plot2}
\end{figure}

Bei der Bestimmung des Elastizitätsmoduls bei beidseitiger Einspannung des eckigen Stabes wird analog der einseitigen Einspannung vorgegangen.
Da allerdings der Stab in zwei Teilen beschrieben wird, werden auch die Werte unabhängig voneinander ausgewertet und geplottet.
Für die erste Hälfte des Stabes, in der Gleichung (\ref{eqn:DurchbiegungL/2}) gültig ist, wird die Hilfsvariable $\eta(x)$ definiert zu
\begin{align}
  \eta(x) = 3L^2x-4x^3.
  \label{eqn:rel2}
\end{align}
Für die andere Hälfte für $x \in [L/2,L]$ wird $\eta(x)$ zu 
\begin{align}
  \eta(x) = 4x^3 - 12L x^2+9L^2x-L^3
  \label{eqn:rel3}
\end{align}
definiert. Werden die Formeln (\ref{eqn:rel2}) und (\ref{eqn:rel3}) analog zu (\ref{eqn:linRegq}) umgestellt und die Steigung $\mu$ mithilfe einer
linearen Regression bestimmt, folgen zwei Werte für das Elastizitätsmodul
\begin{align*}
  \mu_1&= \SI{0.00266 (0.00012)}{\Pa}, & b_1 &= 0.00006 \pm 0.00012, \\
  \mu_2 &= \SI{-0.001 (0.004)}{\Pa}, & b_2 &= 0, \\
  E_1 &= \SI{143.0(6.0)e9}{\Pa}, & E_2 &= \SI{400 (1500)e9}{\Pa}.    
\end{align*}
Der Mittelwert beträgt somit
\begin{align*}
  \overline{E_{\text{\square}}} = \SI{230(50)e9}{\Pa}.
\end{align*}

\subsection{Elastizitätsmodul eines runden Stabes}
\label{sec:elastiRund}

\sloppy
Mit dem runden Stab wird wie mit dem eckigen Stab verfahren. Aus den Maßen des runden Stabes, die aus \autoref{tab:stabRund}
zu entnehmen sind, folgen die Mittelwerte für die Länge $l$ und die Masse $m$ des Stabes
\begin{align*}
  m &= \SI{412.2(0.2)}{\gram}, \\
  l &= \SI{592.0(0.06)}{\mm}. \\
\end{align*}

\sloppy
\begin{table}[H]
  \centering
  \caption{Abmaße des runden Stabes.}
  \label{tab:stabRund}
  \begin{tabular}{c c c}
    \toprule
    l / mm & d / mm & m / g \\
    \midrule
    593 & 10 & 411.8 \\
    592 & 10 & 412.0 \\
    592 & 10 & 412.2 \\
    591 & 10 & 412.2 \\
    592 & 10 & 412.2 \\
    \bottomrule
  \end{tabular}
\end{table}

\subsubsection{Einseitige Einspannung}
\label{subsubsec:rundEinsEing}

Aus \autoref{tab:werte2} sind die Messwerte $D_0(x)$ und $D_{\text G}(x)$ bei einseitiger Einspannung des runden Stabes mit
einem Gewicht von $m_{\text{Last}} = \SI{750}{\gram}$, welches an der Stelle $x= \SI{50}{\cm}$ eingespannt wurde, zu entnehmen.


\sloppy
\begin{table}[H]
  \centering
  \caption{Messung des runden Stabes bei einseitger Einspannung ($m_{\text{Last}} = \SI{750}{\gram}$).}
  \label{tab:werte2}
  \begin{tabular}{c c c}
    \toprule
    x / $\si{\centi\meter} $ & $ D_0(x) / \si{\centi\meter}$ & $D_{\text G}(x) / \si{\milli\meter}$ \\
    \midrule
    3 & 8,13 & 8,07 \\
    6 & 8,21 & 8,00 \\
    9 & 8,27 & 7,98 \\
    12 & 8,34 & 7,87 \\
    15 & 8,41 & 7,72 \\
    18 & 8,45 & 7,44 \\
    21 & 8,57 & 7,32 \\
    24 & 8,63 & 7,11 \\
    27 & 8,74 & 6,84 \\
    30 & 8,81 & 6,60 \\
    33 & 8,92 & 6,32 \\
    36 & 9,01 & 5,99 \\
    39 & 9,09 & 5,65 \\
    42 & 9,10 & 5,37 \\
    45 & 9,10 & 4,93 \\
    48 & 9,14 & 4,64 \\
    \bottomrule
  \end{tabular}
\end{table}

\sloppy
\begin{figure}[H]
  \centering
  \includegraphics[width=0.75\textwidth]{build/plot3.pdf}
  \caption{Messung eines runden Stabes bei einseitiger Einspannung ($\text G = \SI{750}{\gram}$).}
  \label{fig:plot3}
\end{figure}

\sloppy
Die Hilfsvariable $\eta(x)$ ergibt sich nach Gleichung (\ref{eqn:Durchbiegung}) wieder zu Formel (\ref{eqn:eta1}).
%\begin{align}
%  \eta(x) = Lx^2- \frac 13 x^3.
%\end{align}
Nach Gleichung (\ref{eqn:rel1}) ergibt sich analog zu dem eckigen Stab ein linearer Zusammenhang zwischen dem Elastizitätsmodul und
der Variablen $\eta(x)$, wobei der runde Stab die folgenden Werte hat
\begin{align*}
  I_{\bigcirc} &= (\frac{1}{64} \pm 0.0) \si{\kilo\gram\meter^2}, \\
  F_{\bigcirc} &= \SI{4.042 (0.002)}{\newton}. \\
\end{align*}
Die lineare Regression des Types $D(x) = \mu\cdot\eta(x)+b$ liefert die Parameter
\begin{align*}
  \mu &= \SI{0.0455 (0.0007)}{\Pa},\\
  b &= 0.0002 \pm 0.0007, \\
\end{align*}
woraus mit Formel (\ref{eqn:linRegq}) ein Elastizitäsmodul von
\begin{align*}
  E = \SI{164 (2.5)e9}{\Pa}
\end{align*}
folgt.


\subsubsection{Beidseitige Einspannung}
\label{subsubsec:rundBeidEins}
Aus \autoref{tab:werte4} sind die Messwerte für die Auslenkungen $D_0(x)$ und $D_{\text G}(x)$ zu entnehmen, wobei ein Gewicht von
$m_{\text{Last}}=\SI{1550}{\gram}$ bei $x=\SI{27.5}{\cm}$ positioniert wird. \autoref{fig:plot4} ist die grapische Darstellung der Messwerte, die 
jeweils in Bereichen $x \in [0, L/2] $ und 
$x \in [l/2,L]$ unterteilt werden. Es ist außerdem eine lineare Regression für beide Einteilungen
aufzufinden.

\sloppy
\begin{table}[H]
  \centering
  \caption{Messung des runden Stabes bei beidseitiger Auflage ($m_{\text{Last}} = \SI{1550}{\gram}$).}
  \label{tab:werte4}
  \begin{tabular}{c c c}
    \toprule
    x / $\si{\centi\meter} $ & $ D_0(x) / \si{\centi\meter}$ & $D_{\text G}(x) / \si{\milli\meter}$ \\
    \midrule
    3 & 9,93 & 8,67 \\
    6 & 9,95 & 8,59 \\
    9 & 9,97 & 8,44 \\
    12 & 9,86 & 8,30 \\
    15 & 9,82 & 8,12 \\
    18 & 9,67 & 7,85 \\
    21 & 9,60 & 7,74 \\
    24 & 9,52 & 7,63 \\
    27 & 9,36 & 7,51 \\
    30 & 7,50 & 6,63 \\
    33 & 7,38 & 6,56 \\
    36 & 7,26 & 6,54 \\
    39 & 7,17 & 6,51 \\
    42 & 7,04 & 6,49 \\
    45 & 6,95 & 6,49 \\
    48 & 6,68 & 6,48 \\
    51 & 6,72 & 6,53 \\
    54 & 6,51 & 6,60 \\
    \bottomrule
  \end{tabular}
\end{table}

\sloppy
\begin{figure}[H]
  \centering
  \includegraphics[width=0.75\textwidth]{build/plot4.pdf}
  \caption{Messung eines runden Stabes bei beidseitiger Einspannung ($\text G = \SI{1550}{\gram}$).}
  \label{fig:plot4}
\end{figure}


Analog der Berechnung der Elastizitätsmodule des eckigen Stabes bei einseitiger Auflage werden für die lineare Regression die 
Hilfsvariablen mit Gleichung (\ref{eqn:rel2}) und Gleichung (\ref{eqn:rel3}) definiert. Aus der linearen Regression der Form $D(x) = \mu\cdot\eta(x)+b$
folgen die Elastizitätsmodule
\begin{align*}
  \mu_1 &= \SI{0.00376 (0.00023)}{\Pa}, & b_1 &= 0.00115 \pm 0.00023, \\
  \mu_2 &= \SI{0.00587 (0.00031)}{\Pa}, & b_2 &= -0.00037 \pm 0.00031, \\
  E_1 &= \SI{172 (10)e9}{\Pa}, & E_2 &= \SI{110 (6)e9}{\Pa}.\\
\end{align*}
Der Mittelwert beträgt somit
\begin{align*}
  \overline{E_{\bigcirc}} = \SI{149(4)e9}{\Pa}.
\end{align*}