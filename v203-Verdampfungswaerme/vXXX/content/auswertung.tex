\section{Auswertung}
\label{sec:Auswertung}

%Die Tabelle zu den Werten der ersten Aufgabe überstieg die Seitenlänge. Ich habe aber die originalen Daten angefügt und wenn wir
%noch den Plot haben, dann wird das in Ordnung sein.

\subsection{Messung bis 1 bar}
Um die gemittelten Verdampfungswärme zu berechnen muss zuerst der Logarithmus des Dampfdruckes gegen die reziproke absolute Temperatur dargestellt werden. Die gesuchte
Verdampfungswärme kann darauf folgend durch eine Ausgleichsrechnung bestimmt werden. Zur Berechnung dieser wird die Relation
\begin{equation*}
  \label{eqn:druck3}
  \ln\left(\frac{p}{p_0}\right)=- \frac{L}{R} \frac{1}{T}
\end{equation*}
\newline
benutzt, wobei hier $p_0$ den zu Beginn gemessenen Außendruck beschreibt. Dieser hat den Betrag
\begin{equation*}
  p_0 = \SI{993e2}{\pascal}    %993 \cdot \, \mathrm{10^2 Pa}
\end{equation*}

\begin{figure}[H]
  \centering
  \includegraphics{plot1.pdf}
  \caption{Ausgleichsgerade für <1 bar}
  \label{fig:plot1}
\end{figure}
%\newline
\noindent
Die Ausgleichsgerade hat die Form
\begin{equation*}
  y = mx + b
\end{equation*}
%\newline
Die Parameter und ihre Unsicherheiten wurden mit Python berechnet und haben die ungefähren Werte
\begin{equation*}
  m \approx (-44.22\pm 0.4924)
\end{equation*}
und
\begin{equation*}
  b \approx (11.67\pm 0.1525)
\end{equation*}
%\newline
Wird nun die Formel der Ausgleichsgeraden mit (\ref{eqn:expo}) zusammengeführt, ergibt sich daraus
\begin{equation*}
  m=-\frac{L}{R}
\end{equation*}
%\newline
Daraus folgt dann für die Verdampfungswärme L
\begin{equation*}
  L = - m \cdot R
\end{equation*}
wobei $R$ hier mit $R=\SI{8,314}{\joule\per\kelvin\mole}$ als allgemeine Gaskonstante gegeben ist.
%\newline
Eingesetzt ergibt sich für $L$ dann
\begin{equation*}
  L = \SI{367,66}{\joule\per\mole}
\end{equation*}

\noindent
Die äußere Verdampfungswärme $L_{\text{a}}$ beschreibt hierbei die Energie, die benötigt wird, um das Volumen eines Mols der Flüssigkeit auf das Volumen eines Mols des Gases zu
erhöhen. Hierfür muss die Volumenarbeit $W = pV$ erbracht werden. Diese wird mit der idealen Gasgleichung gleichgesetzt
\begin{equation*}
  W = L_{\text{a}} = p\cdot V = R \cdot T
\end{equation*}
woraus für eine Temperatur von T $= \SI{373}{\kelvin}$ 
\begin{equation*}
  L_{\text{a}} = \SI{3,101}{\joule\per\mole}  %3,101 \cdot 10^3 \mathrm{\frac{J}{mol}}
\end{equation*}
folgt.
\newline
Die, um die molekularen Bindungskräfte zu überwinden, benötigte innere Energie $L_{\text{i}}$ ergibt sich daraus zu
\begin{equation*}
  L_{\text{i}} = L - L_{\text{a}} = \SI{-2,733}{\joule\per\mole}  %-2.733,34 \mathrm{\frac{J}{mol}}
\end{equation*}
%\newline
Aufgrund dieses, wegen seines Vorzeichens offensichtlich falschen, Wertes wird auf eine division durch die Avogadrokonstante und anschließende Angabe in eV verzichtet.

\subsection{Messung von 1 bis 15 bar}
Im zweiten Schritt der Auswertung soll nun die Temperaturabhängigkeit der Verdampfungswärme $L$ bestimmt werden. Hierzu wird die Clausius-Clapeyronsche Gleichung nach $L$ umgestellt.
\begin{equation}
  L = T(V_{\text{D}} - V_{\text{F}})\frac{\mathrm{d}p}{\mathrm{d}T}   %\frac{\mathrm{d}p}{\mathrm{d}T}
  \label{eqn:verdl}
\end{equation}
%\newline
$V_{\text{F}}$ kann hierzu weiterhin vernachlässigt werden. $V_{\text{D}}$ hingegen kann in diesem Bereich des Drucks nicht mehr über die ideale Gasgleichung bestimmt werden.
Eine genauere Nährung ist gegeben als 
\begin{equation*}
  \left(p+\frac{a}{V^2}\right)V = RT \, \, \mathrm{mit} \, \, a =  %0,9 \mathrm{\frac{J \cdot m^3}{Mol^2}} 
\end{equation*}
\begin{equation*}
  \Leftrightarrow V_{\text{D}} = \frac{RT}{2p} \pm \sqrt{\frac{R^2T^2}{4p^2}-\frac{a}{p}}
\end{equation*}
%\newline
Damit wird \ref{eqn:verdl} zu
\begin{equation}
  L = \frac{T}{p} \left(\frac{RT}{2} \pm \sqrt{\frac{R^2T^2}{4}-ap}\right)\frac{\mathrm{d}p}{\mathrm{d}T}. %\frac{\mathrm{d}p}{\mathrm{d}T}
  \label{eqn:verdl2}
\end{equation}
Für die Bestimmung von $\frac{\mathrm{d}p}{\mathrm{d}T}$ muss ein Ausgleichspolynom 3.Grades für die Messwerte der Temperatur $T$ und des Drucks $p$ erstellt werden. 
Diese Messwerte sind in \autoref{tab:werte1} aufgetragen.

\begin{table}[H]
  \centering
  \caption{Gemessene Messwerte der Verdampfungswärme.}
  \label{tab:werte1}
  \begin{tabular}{c c}
    \toprule
    p $\,/\,\si{\bar}$ & t $\,/\, \si{\celsius}$ \\
    \midrule
    1 & 116 \\
    2 & 133 \\
    3 & 141 \\
    4 & 149 \\
    5 & 156 \\
    6 & 163 \\
    7 & 168 \\
    8 & 173 \\
    9 & 176 \\
    10 & 181 \\
    11 & 185 \\
    12 & 188 \\
    13 & 191 \\
    14 & 194 \\
    15 & 197 \\
    \bottomrule
  \end{tabular}
\end{table}

\begin{figure}
  \centering
  \includegraphics{plot2.pdf}
  \caption{Ausgleichsgerade für >1 bar}
  \label{fig:plot2}
\end{figure}
\newpage
Das Ausgleichspolynom hat die Form
\begin{center}
  $p(T)= A \cdot T^3 + b \cdot T^2 + c \cdot T + d$ \\
  $\frac{\mathrm{d}p}{\mathrm{d}T} = 3a \cdot T^2 + 2b \cdot T + c$ \\
  $a = 1,0540 \cdot 10^{-5}$  \\
  $b = -1,1760 \cdot 10^{-2}$ \\
  $c = 4,4246$ \\
  $d = -5,6107 \cdot 10^2$ \\
\end{center}
%\newline
$p(T)$ und $\frac{\mathrm{d}p}{\mathrm{d}T}$ werden nun in Formel (\ref{eqn:verdl2}) eingesetzt:
\begin{equation*}
  L = \frac{3A \cdot T^3 + 2b \cdot T^2 + cT}{A \cdot T^3 + b \cdot T^2 + c \cdot T + d} \left(\frac{RT}{2} \pm \sqrt{\frac{R^2T^2}{4}-a\cdot(A \cdot T^3 + b \cdot T^2 + c \cdot T + d)}\right)
\end{equation*}