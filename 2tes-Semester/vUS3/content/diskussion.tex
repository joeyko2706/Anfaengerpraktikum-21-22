\section{Diskussion}
\label{sec:Diskussion}

Beim Aufbau an sich ist anzumerken, dass die Flußgeschwindigkeit durch den kleinen Radius des Schlauches eventuell beinträchtigt werden könnte.
Außerdem konnte der Schlauch nicht an das Acryl mit den drei einzustellenden Winkeln gedrückt werden, da die Kunststoffplatte nicht 
unter den Schlauch passte und somit nicht für bessere Fixierung druntergeschoben werden konnte.\newline
Im ersten Teil des Versuches konnten genaue Messwerte erfasst werden und eine grafische Auswertung dieser ergibt die zu erwartende lineare Abhängigkeit
zwischen dem Quotienten $\sfrac{\upDelta\nu}{\cos(\alpha)}$ und der Fließgeschwindigkeit $v$. \newline
Der zweite Teil des Versuches ergab jedoch keine gebrauchbaren Messwerte. Dies liegt vermutlich an der Einstellung des Ultraschall-Doppler Generators.
Es war nicht möglich den Fehler, der die Werte so stark verzerrte, zu finden. Es war zu erwarten, dass die Fließgeschwindigkeit mit der Tiefe zunimmt
und ein Maximum erreicht, wenn der Ultraschall auf der Tiefe der zu untersuchenden Flüssigkeit angekommen ist. Danach hätte die gemessen 
Fließgeschwindigkeit wieder abnehmen sollen.
In unserem Versuch ergaben sich keine Werte und es gab kein deutlich abzulesendes Maximum. Es ist deshalb zu sagen, dass der zweite Teil des Versuches
nicht auszuwerten ist.\newline
Eine mögliche Fehlerquelle ist das Programm \textit{Flow View}, in dem die angezeigten Werte für die Maximal- und Minimalfrequenz $f_{\text{max}},\, \,f_{\text{mean}}$
starken Schwankungen unterlagen. Es war demnach nicht möglich einen genauen Wert abzulesen, wodurch die dadurch bestimmte Frequenzdifferenz ebenfalls
hohen Schwankungen unterliegt.
Eine weitere Fehlerquelle könnte die Menge des verwendetem Ultraschallgel sein. Da hier jedoch mehrere Male mit unterschiedlichen Mengen versucht wurde, Messwerte
zu erstellen und die Qualität nicht besser wurde, ist davon auszugehen, dass zumindest diese Fehler ausgebessert wurden.
Die Strömungsgeschwindigkeit, die von der Pumpe erzeugt wurde, unterlag ebenfalls hohen Schwankungen. Es war nicht möglich eine genaue Pumpleistung
einzustellen, da bereits durch Erschütterungen des Tisches, auf dem die Pumpe stand, zu einer anderen Pumpleistung führte.
All diese Ungenauigkeiten führten zu schlechteren Messwerten. Jedoch bleibt immer noch offen, weshalb die Messwerte im zweiten Versuchsteil so ungebrauchbar sind.