\section{Diskussion}
\label{sec:Diskussion}
In \autoref{subsec:emissionsspektrum} wurde die Kupfer-Röntgenröhre hinsichtlich des Emissionsspektrums ausgewertet. Dabei kamen die in \autoref{tab:Disk1} eingetragenen Werte heraus.

\begin{table}[H]
    \caption{Theoretische und experimentell bestimmte Werte des Absorptionsspektrums.}
    \label{tab:Disk1}
    \centering
    \begin{tabular}{c c c}
        \toprule
         & $K_\alpha$ & $K_\beta$\\
        \midrule
        Maximum & $22,5 ~ 2\vartheta$ & $20,2 ~ 2\vartheta$ \\
        theoretische Energie & $\SI{8047,823}{\eV}$ & $\SI{8905,413}{\eV}$ \\
        gemessene Energie & $\SI{8043,355}{\eV}$ & $\SI{8914,204}{\eV}$ \\
        Abweichung & $0,056 \%$ & $0,0987 \%$ \\
        Halbwertsbreite & $\SI{0,42}{\degree}$ & $\SI{0,451}{\degree}$ \\
        \bottomrule
    \end{tabular}
\end{table}
Da die Abweichungen der experimentell bestimmten Werte von den theoretischen Werten \cite{NistXray} \cite{Abschirmkonstanten} sehr gering sind, ist die Theorie damit verifiziert worden.\newline
Die Emissionsspektren der verschiedenen Stoffe sind in \autoref{subsec:absorbis} ausgewertet worden, wobei die in \autoref{tab:Disk2} eingetragenen Werte bestimmt wurden.
\begin{table}[H]
    \caption{Theoretische und experimentell bestimmte Werte der Absorptionsspektren verschiedener Stoffe.}
    \centering
    \label{tab:Disk2}
    \begin{tabular}{c| c c c c c c}
        \toprule
        Element  & $E$ / keV & $E_{\text{Theorie}}$ / keV & $\upDelta E$ & $\sigma$ & $\sigma_{\text{Theorie}}$ & $\upDelta \sigma$ \\
        \midrule
        Brom      & $13,185$  & $13,470$  & $2,12 \%$ & $3,84$ & $3,55$ & $8,37   ~\%$ \\
        Gallium   & $10,179$  & $10,368$  & $1,82 \%$ & $3,61$ & $3,41$ & $5,94  ~\%$ \\
        Strontium & $15,709$  & $16,107$  & $2,47 \%$ & $3,99$ & $3,61$ & $10,56 ~\%$ \\
        Zink      & $9,551$   & $9,660$   & $1,13 \%$ & $3,55$ & $3,37$ & $5,36  ~\%$ \\
        Zirkonium & $17,215$  & $17,995$  & $4,34 \%$ & $4,09$ & $3,65$ & $12,20 ~\%$ \\
        \bottomrule
    \end{tabular}
\end{table}
Auch hier sind die Abweichungen sehr gering und es kann gesagt werden, dass die Theorie bestätigt wurde. Die Abweichung für Strontium und Zirkonium sind die einzigen, die $10 \%$ übersteigen. Um
dieser Abweichung entgegenzuwirken, können mehr Messwerte in einem kleineren Winkelabstand gemessen werden, sofern das mit der Apparatur möglich ist. Es könnte außerdem die
Messzeit von $\SI{20}{\second}$ erhöht werden, um genauere Messwerte zu bekommen.\newline
Die in \autoref{subsec:moseley} mit den Messwerten bestimmte Rydberg-Energie hat die folgenden Werte,
\begin{align*}
    R_\infty &= \SI{12,86(0,06)}{\eV}, \\
    R_{\infty, \text{Theorie}} &= \SI{13,6}{\eV}, \\
    \text{Abweichung} &= (5,5 \pm 0,5) ~\%.
\end{align*}
Auch diese Abweichungen sind als gering einzuschätzen und die Theorie wurde auch hier bestätigt. Die Abweichung kann mit mehr Proben, mit denen das Moseley'sche Gesetz
überprüft werden kann, und mit den oben bereits genannten Methoden herabgesetzt werden.

\printbibliography{}

\section*{Anhang}
\label{sec:anhang}

\begin{table}[H]
    \caption{Originale Messdaten des Bromabsorbers.}
    \centering
    \label{tab:origDaten2}
    \begin{tabular}{c c}
        \toprule
        $2~\vartheta$ & $R(\SI{35}{\kilo\volt})$ Imp/s \\
        \midrule
        12.7  &  8.0  \\
        12.8  &  10.0  \\
        12.9  &  11.0  \\
        13.0  &  9.0  \\
        13.1  &  13.0  \\
        13.2  &  17.0  \\
        13.3  &  19.0  \\
        13.4  &  20.0  \\
        13.5  &  24.0  \\
        13.6  &  20.0  \\
        13.7  &  21.0  \\
        \bottomrule
    \end{tabular}
\end{table}

\begin{table}[H]
    \caption{Originale Messdaten des Galliumabsorbers.}
    \centering
    \label{tab:origDaten3}
    \begin{tabular}{c c}
        \toprule
        $2~\vartheta$ & $R(\SI{35}{\kilo\volt})$ Imp/s \\
        \midrule
        16.7  &  40.0  \\
        16.8  &  41.0  \\
        16.9  &  41.0  \\
        17.0  &  41.0  \\
        17.1  &  46.0  \\
        17.2  &  57.0  \\
        17.3  &  64.0  \\
        17.4  &  71.0  \\
        17.5  &  77.0  \\
        17.6  &  78.0  \\
        17.7  &  77.0  \\
        \bottomrule
    \end{tabular}
\end{table}

\begin{table}[H]
    \caption{Originale Messdaten des Strontiumabsorbers.}
    \centering
    \label{tab:origDaten4}
    \begin{tabular}{c c}
        \toprule
        $2~\vartheta$ & $R(\SI{35}{\kilo\volt})$ Imp/s \\
        \midrule
        10.5  &  34.0  \\
        10.6  &  36.0  \\
        10.7  &  35.0  \\
        10.8  &  35.0  \\
        10.9  &  43.0  \\
        11.0  &  60.0  \\
        11.1  &  79.0  \\
        11.2  &  91.0  \\
        11.3  &  101.0  \\
        11.4  &  100.0  \\
        11.5  &  107.0  \\
    \end{tabular}
\end{table}

\begin{table}[H]
    \caption{Originale Messdaten des Zinkabsorbers.}
    \centering
    \label{tab:origDaten5}
    \begin{tabular}{c c}
        \toprule
        $2~\vartheta$ & $R(\SI{35}{\kilo\volt})$ Imp/s \\
        \midrule
        18.1  &  49.0  \\
        18.2  &  49.0  \\
        18.3  &  48.0  \\
        18.4  &  53.0  \\
        18.5  &  65.0  \\
        18.6  &  73.0  \\
        18.7  &  84.0  \\
        18.8  &  93.0  \\
        18.9  &  93.0  \\
        19.0  &  90.0  \\
        19.1  &  92.0  \\
    \end{tabular}
\end{table}

\begin{table}[H]
    \caption{Originale Messdaten des Zirkoniumabsorbers.}
    \centering
    \label{tab:origDaten6}
    \begin{tabular}{c c}
        \toprule
        $2~\vartheta$ & $R(\SI{35}{\kilo\volt})$ Imp/s \\
        \midrule
        9.3  &  57.0  \\
        9.4  &  63.0  \\
        9.5  &  68.0  \\
        9.6  &  69.0  \\
        9.7  &  70.0  \\
        9.8  &  82.0  \\
        9.9  &  87.0  \\
        10.0  &  111.0  \\
        10.1  &  127.0  \\
        10.2  &  133.0  \\
        10.3  &  137.0  \\
    \end{tabular}
\end{table}