\section{Auswertung}
\label{sec:Auswertung}

Die Messwerte des Versuches, wie er in \autoref{sec:Durchführung} beschrieben wurde, werden nun ausgewertet und mit den theoretischen und literarischen Werten verglichen.

\subsection{Auswertung der Kennlinienschar einer Hochvakuumdiode}
\label{subsec:kennL}
Zu Beginn der Auswertung wird durch Variation der eingsetellten Heizleistung $P_{\text H}$ eine Kennlinienschar der Hochvakuumdiode erstellt. Dazu werden die Messwerte grafisch dargestellt und mithilfe einer grafischen Auswertung der Sättigungsstrom bestimmt.
Die Messwerte werden mit den Pythonerweiterungen Numpy \cite{numpy} und Matplotlib \cite{matplotlib} erstellt.
In \autoref{fig:plot1} sind die Kennlinien der vier geringsten Leistungen abgebildet. Die originalen Messdaten sind dabei in \autoref{sec:anhang} zu finden.

\begin{figure}[H]
  \centering
  \includegraphics{plot1.pdf}
  \caption{Grafische Auswertung der Messwerte für die vier geringsten Leistungen.}
  \label{fig:plot1}
\end{figure}

\noindent
Der Sättigungsstrom wurde in der \autoref{fig:plot1} durch eine horizontale Linie eingetragen. 
Zu erkennen ist, dass eine grafische Auswertung für eine Leistung von $\SI{10,35}{\watt}$ keine genauen Ergebnisse liefert. Deshalb wird, genauso wie für eine Leistung von $\SI{13,75}{\watt}$ verfahren. \newline
Es werden die Messwerte des Stromes $[\si{\milli\ampere}]$ gegenüber der Messwerte der Spannung $[\si{\volt}]$ dargestellt und eine Regression dritten Grades an die Messwerte mithilfe der Pythonerweiterung
Scipy \cite{scipy} angelegt. Der Sättigungsstrom ergibt sich als den zweifachen Wert des Spannungswertes des Wendepunktes.
Die grafische Auswertung durch eine Regression dritten Grades ist in \autoref{fig:plot2} zu finden.

\begin{figure}[H]
  \centering
  \includegraphics{plot2.pdf}
  \caption{Grafische Auswertung der beiden größten Leistungen mithilfe einer Regression dritten Grades.}
  \label{fig:plot2}
\end{figure}

\noindent
Die Regression dritten Grades der Form
\begin{align*}
  y = ax^3+bx^2+cx+d  
\end{align*}
liefert die in \autoref{tab:reg3} eingetragenen Parameter.

\begin{table}[H]
  \caption{Werte der Regression vom Grad drei.}
  \label{tab:reg3}
  \centering
  \begin{tabular}{c c c}
      \toprule
      Parameter & $\SI{10,35}{\watt}$ & $\SI{13,75}{\watt}$ \\
      \midrule
      $a \cdot 10^{-7}$ & $-2,45 \pm 0,09$ & $-1,17 \pm 0,09$ \\
      $b \cdot 10^{-5}$ & $6,39 \pm 0,24$ & $5,98 \pm 0,34$ \\
      $c \cdot 10^{-3}$ & $2,97 \pm -0,18$ & $4,1 \pm 0,4$ \\
      $d$ & $-0,009 \pm 0,004$ & $-0,029 \pm 0,010$ \\
      \bottomrule
    \end{tabular}
\end{table}

\noindent
Es ergeben sich somit die beiden Wendepunkte zu
\begin{align*}
  WP_{10,35} &= (87 \pm 4 | 0,57 \pm 0,06), \\
  WP_{13,75} &= (171 \pm 16 | 1,84 \pm 0,34). \\
\end{align*}
Somit folgen die Sättigungsströme zu den in \autoref{tab:saettis} eingetragenen Werten.

\begin{table}[H]
  \caption{Ausgemessene Sättigungsströme.}
  \label{tab:saettis}
  \centering
  \begin{tabular}{c c}
      \toprule
      Heizleistung $P_{\text H}$ / $\si{\watt}$ & Sättigungsströme $I_{\text S}$ / $\si{\milli\ampere}$ \\
      \midrule
      $5,7$   & $0,047$, \\
      $7,0$   & $0,11$, \\
      $8,1$   & $0,284$, \\
      $10,35$ & $1.14 \pm 0.12$, \\
      $13,75$ & $3.7 \pm 0.7$, \\
      \bottomrule
    \end{tabular}
\end{table}

\subsection{Gültigkeitsbereich des Langmuir-Schotky'schen Raumladungsgesetz}
\label{subsec:langmuirSchottky}

Es wird im folgenden der Gültigkeitsbereich der Langmuir-Schottky'schen Gleichung untersucht.
Aus Gleichung \eqref{eqn:Raumladung} folgt ein Zusammenhang zwischen dem Strom $I$ und der Spannung $U$ der Form $I=b\cdot U^m$, woraus nach umstellen ein linearer Zusammenhang der Form
\begin{align}
  \label{eqn:physLinReg}
  \log\left(\frac{I}{I_0}\right) = m \cdot \log\left(\frac{U}{U_0}\right) +b
\end{align}
wird. Um die Gültigkeit des Langmuir-Schotky'schen Raumladungsgesetzes zu überprüfen werden nun die Logarithmen des Stromes gegenüber denen der Spannung grafisch aufgetragen und
eine lineare Regression der Form
\begin{align}
  \label{eqn:linReg}
  y=m\cdot x+b
\end{align}
angelegt. Die grafische Auswertung ist \autoref{fig:plot3} zu entnehmen.

\begin{figure}[H]
  \centering
  \includegraphics{plot3.pdf}
  \caption{Raumladungsbereich der $W_{\text H} = 13,75$ Kennlinie.}
  \label{fig:plot3}
\end{figure}

\noindent
Es ergeben sich die folgenden Parameter
\begin{align*}
  m = 1,437 \pm 0,0146, \quad
  b = -6,8027 \pm 0,0670.
\end{align*}

\subsection{Auswertung des Anlaufstromgebietes}
\label{subsec:anlaufstrom}

Mithilfe einer Auswertung des Anlaufstromgebietes, ist es möglich die Kathodentemperatur zu bestimmen.
Aus der Gleichung \eqref{eqn:j_Anlauf} folgt ein exponentieller Zusammenhang zwischen der Heizspannung und des Anlaufstromes.
Werden die Messwerte halblogarithmisch dargestellt, sollte sich ein linearer Zusammenhang zwischen der Spannung und dem Strom ergeben.
Die halblogarithmische, grafische Auswertung ist \autoref{fig:plot4} zu entnehmen. Die grün markierten Messwerte sind dabei die Messwerte, bei denen der
Messwertebereich am Amperemeter umgestellt wurde.

\begin{figure}[H]
  \centering
  \includegraphics{plot4.pdf}
  \caption{Halblogarithmische Darstellung der Messwerte des Anlaufstromgebietes.}
  \label{fig:plot4}
\end{figure}

\noindent
Die lineare Regression der Form analog zu Formel \eqref{eqn:linReg} ergibt sich analog zu der in \autoref{subsec:langmuirSchottky} aus der Gleichung \eqref{eqn:j_Anlauf} zu einer Steigung von
\begin{align}
  \label{eqn:steigung}
  m = -\frac{e}{kT}.
\end{align}
Die Parameter ergeben sich zu
\begin{align*}
  m =  \SI{-4,0503 \pm 0,4855}{\per\volt}, \quad
  b = -15,0008 \pm 0,2698.
\end{align*}
\noindent
Es folgt also eine Kathodentemperatur von
\begin{align*}
  \SI{2865,095 \pm 343,460}{\kelvin}.
\end{align*}

\subsection{Kathodentemperatur der Kennlinienschar}
\label{subsec:kathodenTempSchar}

Im nächsten Auswertungsschritt wird mithilfe der aufgenommenen Messwerte zur Kennlinienschar zuerst die Kathodentemperatur berechnet und damit
die Austrittsarbeit bestimmt. Die Temperatur der Kathode folgt aus Gleichung \eqref{eqn:Leistung}, wenn diese nach $T$ umgestellt wird zu 
\begin{align}
  T = \left(\frac{I_{\text H}\cdot U_{\text H} - N_{\text{WL}}}{f\eta\sigma}\right)^{\frac 14}.
\end{align}
Dabei ist die verwendete Wärmeleistung der Kathode $N_{\text{WL}} = \SI{0,95}{\watt}$, die emittierte Fläche $f=\SI{0,32}{\centi\meter\squared}$
und der Emissionsgrad $\eta = 0,28$. Die Kathodentemperaturen zu den jeweiligen Heizleistungen sind in \autoref{tab:heizis} eingetragen.

\begin{table}[H]
  \caption{Ausgerechnete Kathodentemperaturen zu den jeweiligen Heizleistungen $P_{\text H}$.}
  \label{tab:heizis}
  \centering
  \begin{tabular}{c c}
      \toprule
      $P_{\text H}$ / $\si{\watt}$ & $T \,/\, \si{\kelvin}$\\
      \midrule
      $13,75$ & $2229,3641$ \\
      $10,35$ & $2073,9679$ \\
      $8,1$   & $1936,8526$ \\
      $7,0$   & $1857,6287$ \\
      $5,7$   & $1748,6121$ \\
      \bottomrule
    \end{tabular}
\end{table}

\subsection{Auswertung der Austrittsarbeit von Wolfram mithilfe der Richardson-Gleichung}
\label{subsec:richard}

Mit den ausgewerteten Sättigungsströmen $I_{\text S}$ wird nun die Austrittsarbeit $\phi$ für Wolfram berechnet. Mithilfe der Relation
$j_S= \sfrac{I_{\text S}}{f}$ folgt aus Gleichung \eqref{eqn:J_S} der folgene Ausdruck für die Austrittsarbeit $\phi$, dessen Werte in \autoref{tab:austrittis} eingetragen sind,
\begin{align*}
  \phi = -\frac ke T\cdot\ln\left(\frac{I_{\text S}\cdot h^3}{4\pi fem_0k^1T^1}\right).
\end{align*}

\begin{table}[H]
  \caption{Ausgerechnete Austrittsarbeiten $\phi$ für die verschiedenen Heizleistungen $P_{\text H}$ mit den jeweiligen Sättigungsströmen $I_{\text S}$.}
  \label{tab:austrittis}
  \centering
  \begin{tabular}{c c c}
      \toprule
      $P_{\text H} \,/\, \si{\watt}$ & $I_{\text S} \,/\, \si{\milli\ampere}$ & $\phi \,/\, \si{\eV}$\\
      \midrule
      $13,75$ & $3,7 \pm 0,7$   & $4,740 \pm 0,035$ \\
      $10,35$ & $1,14 \pm 0,12$ & $4,593 \pm 0,018$ \\
      $8,1$   & $0,284$         & $4,499$ \\
      $7,0$   & $0,11$          & $4,453$ \\
      $5,7$   & $0,047$         & $4,302$ \\
      \bottomrule
    \end{tabular}
\end{table}

\noindent
Mit diesen Werten ergibt sich ein Mittelwert der Austrittsarbeit $\overline \phi$ zu
\begin{align*}
  \overline \phi = \SI{4,517 \pm 0,008}{\eV}.
\end{align*}