\section{Auswertung}
\label{sec:Auswertung}

Zu Beginn der Auswertung wird durch Variation eine Kennlinienschar der Hochvakuumdiode erstellt. Dazu werden die Messwerte grafisch dargestellt und mithilfe einer grafischen Auswertung der Sättigungsstrom bestimmt.
Die Messwerte werden mit den Pythonerweiterungen Numpy \cite{numpy} und Matplotlib \cite{matplotlib} erstellt.
In \autoref{fig:plot1} sind die Kennlinien der vier geringsten Leistungen abgebildet. Die originalen Messdaten sind dabei in \autoref{sec:anhang} zu finden.

\begin{figure}
  \centering
  \includegraphics{plot1.pdf}
  \caption{Grafische Auswertung der Messwerte für die vier geringsten Leistungen.}
  \label{fig:plot1}
\end{figure}

\noindent
Der Sättigungsstrom wurde in der \autoref{fig:plot1} durch eine horizontale Linie eingetragen. 
Zu erkennen ist, dass eine grafische Auswertung der Kennlinie für eine Leistung von
$\SI{10,35}{\watt}$ keine genauen Ergebnisse liefert. Deshalb wird, genauso wie für eine Leistung von $\SI{13,75}{\watt}$ verfahren.
Es werden die Messwerte des Stromes $[\si{\milli\ampere}]$ gegenüber der Messwerte der Spannung $[\si{\volt}]$ dargestellt und eine Regression dritten Grades an die Messwerte mithilfe der Pythonerweiterung
Scipy \cite{scipy} angelegt. Der Sättigungsstrom ergibt sich als den zweifachen Wert des Spannungswertes des Wendepunktes.
Die grafische Auswertung durch eine Regression dritten Grades ist in \autoref{fig:plot2} zu finden.

\begin{figure}
  \centering
  \includegraphics{plot2.pdf}
  \caption{Grafische Auswertung der beiden größten Leistungen mithilfe einer Regression dritten Grades.}
  \label{fig:plot2}
\end{figure}

\noindent
Die Regression dritten Grades der Form
\begin{align*}
  y = ax^3+bx^2+cx+d  
\end{align*}
liefert die in \autoref{tab:reg3} eingetragenen Parameter.

\begin{table}[H]
  \caption{Werte der Regression vom Grad drei.}
  \label{tab:reg3}
  \centering
  \begin{tabular}{c c c}
      \toprule
      Parameter & $\SI{10,35}{\watt}$ & $\SI{13,75}{\watt}$ \\
      \midrule
      $a \cdot 10^{-7}$ & -2,4492 & -1,1691 \\
      $b \cdot 10^{-5}$ & 6,3927 & 5,9830 \\
      $c \cdot 10^{-3}$ & 2,9691 & 4,1352 \\
      $d \cdot 10^{-3}$ & -9,1250 & -29,0294 \\
      \bottomrule
    \end{tabular}
\end{table}

\noindent
Es ergeben sich somit die beiden Wendepunkte zu
\begin{align*}
  WP_{10,35} &= (87,0047 | 0,5718), \\
  WP_{13,75} &= (170,593 | 1,8372). \\
\end{align*}
Somit folgen die Sättigungsströme zu den in \autoref{tab:saettis} eingetragenen Werten.

\begin{table}[H]
  \caption{Ausgemessene Sättigungsströme.}
  \label{tab:saettis}
  \centering
  \begin{tabular}{c c}
      \toprule
      Heizleistung $W_{\text H}$ / $\si{\watt}$ & Sättigungsströme $I_{\text S}$ / $\si{\milli\ampere}$ \\
      \midrule
      $W_{\text H} = 5,7$ & $I_{\text S} = 0,047$, \\
      $W_{\text H} = 7,0$ & $I_{\text S} = 0,11$, \\
      $W_{\text H} = 8,1$ & $I_{\text S} = 0,284$, \\
      $W_{\text H} = 10,35$ & $I_{\text S} = 1,1436$, \\
      $W_{\text H} = 13,75$ & $I_{\text S} = 3,6744$, \\
      \bottomrule
    \end{tabular}
\end{table}

\subsection{Gültigkeitsbereich des Langmuir-Schotky'schen Raumladungsgesetz}
\label{subsec:langmuirSchottky}

Aus Gleichung [REFERENZ] folgt ein Zusammenhang zwischen dem Strom $I$ und der Spannung $U$ der Form $I=b\cdot U^m$, woraus nach umstellen ein linearer Zusammenhang der Form
\begin{align}
  \log\left(\frac{I}{I_0}\right) = m \cdot \log\left(\frac{U}{U_0}\right) +b.
\end{align}
Um die Gültigkeit des Langmuir-Schotky'schen Raumladungsgesetz zu überprüfen werden nun die Logarithmen des Stromes gegenüber dem der Spannung grafisch aufgetragen und
eine lineare Regression der Form
\begin{align*}
  y=m\cdot x+b
\end{align*}
angelegt. Die grafische Auswertung ist \autoref{fig:plot3} zu entnehmen.

\begin{figure}
  \centering
  \includegraphics{plot3.pdf}
  \caption{Raumladungsbereich der $W_{\text H} = 13,75$ Kennlinie.}
  \label{fig:plot3}
\end{figure}

\noindent
Es ergeben sich die folgenden Parameter

\begin{figure}
  \centering
  \includegraphics{plot4.pdf}
  \caption{Plot.}
  \label{fig:plot4}
\end{figure}
