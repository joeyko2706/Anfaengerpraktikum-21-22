\section{Zielsetzung}
\label{sec:Zielsetzung}
Ziel dieses Versuches ist es, den Elastizitätsmodul eines Metalles durch Biegen eines Metallstabes zu bestimmen. 

\section{Theorie}
\label{sec:Theorie}
Der Elastizitätsmodul $E$ ist eine Materialkonstante, die die Verformung eines Körpers unter einer Normalpannung $\sigma$ beschreibt.
Aus der durch die Normalpannung resultierenden Längenänderung $\Delta L$ folgt mit dem Hookschen Gesetz der Zusammenhang
\begin{align}
    \label{eqn:Elastizitaetsmodul}
    \sigma = E \cdot \frac{\Delta L}{L}.
\end{align}
Um einen Metallstab zu verbiegen, muss an ihm ein äußeres Drehmoment wirken. Die Biegung bewirkt, dass sich die oberen Schichten im Metallstab außeinander strecken und die Unteren zusammen
stauchen. Der Bereich in der Mitte des Stabes, der keine Veränderung seiner Länge erfährt, wird neutrale Faser genannt.
Durch solch eine Deformation des Stabes entsteht ein inneres Drehmoment, das dem äußeren entgegengesetzt wirkt und den gleichen Betrag hat.
Die Drehmomente sind gegeben durch
\begin{align*}
    M_{\text{F}} &= F(L-x) \\
    M_{\sigma} &= \int_{Q} y\sigma(y)\text{dq}, \\
\end{align*}
wobei $F$ dabei die von außen wirkende Kraft, $Q$ der Querschnitt des Stabes und $y$ der Abstand des Flächenelements $\text{dq}$ zur neutralen Faser ist.
Für die Durchbiegung eines einseitig befestigten Stabes ergibt sich somit
\begin{align}
    \label{eqn:Durchbiegung}
    D(x) = \frac{F}{2EI} \cdot\left(Lx^2 - \frac{x^3}{3}\right).
\end{align}
Hierbei beschreibt $I$ das Flächenträgheitsmoment, $L$ die Länge des Stabes und $x$ die Entfernung des Messpunktes zum Einspannpunkt.
Ist der Stab beidseitig befestigt, sodass die Kraft auf ihn in seiner Mitte wirkt, ergibt sich für $0 \leq x \leq L/2$
\begin{align}
    \label{eqn:DurchbiegungL/2}
    D(x) = \frac{F}{48EI}\cdot \left(3L^2 x - 4x^3\right).
\end{align}
Demnach ergibt sich für $L/2 \leq x \leq L$
\begin{align}
    \label{eqn:DurchbiegungL}
    D(x) = \frac{F}{48EI}\cdot \left(4x^3 -12Lx^2 +9L^2 x - L^3\right).
\end{align}
Für einen Stab mit quadratischem Querschnitt und der Seitenlänge $a$ ist das Flächenträgheitsmoment gegeben durch
\begin{align}
    \label{eqn:quadrat}
    I_{\text{\square}} = \frac{a^4}{12}.
\end{align}
Das Flächenträgheitsmoment eines Stabes mit kreisförmigem Querschnitt und Durchmesser $d$ ist durch
\begin{align}
    \label{eqn:kreis}
    I_{\bigcirc} = \frac{\pi d^4}{64}
\end{align}
gegeben.
