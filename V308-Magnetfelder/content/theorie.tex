\usepackage{unicode-math}\usepackage{siunitx}

\section{Theorie}
\label{sec:Theorie}
Theoretische Grundlagen für diesen Versuch sind das Biot-Savartsche Gesetz, Stoffmagnetismus mit besonderer Rücksicht auf Ferrmagnetismus,
der Hall-Effekt (und Magnetfelder in Materie).

Bewegte elektrische Ladungsträger erzeugen magnetische Felder, die durch die beiden Größen $\vec H$ und $\vec B$ beschrieben werden können.
Dabei entspricht $\vec H$ der magnetischen Feldstärke, die Betrag und Richtung des Magnetfeldes beschreibt, und $\vec B$ die magnetische Flussdichte,
die zusätzlich noch die Permeabilität $\mu=\mu_0 \cdot\mu_r$ berücksichtigt. Dabei ist $\mu_0=4\pi 10^{-7} [\frac{\text{Vs}}\text{{Am}}]$ die
magnetische Feldkonstante und $\mu_r$ die vom Material abhängige Permeabilität. 
Die magnetische Feldstärke lässt sich zum Beispiel mit dem Biot-Savartschen Gesetz (1) quantifizieren:

\begin{equation}
    equation = \vec B = \frac{\mu_0\cdot I}{4\pi}\int_{\text{Leiter}}\frac{d\vec s\times\vec l}{r^3}.
\end{equation}

%Mit dem durch den Leiter fließenden Strom I und dem infinitesimalen Leiterstück $d\vec l$ lässt sich som

Durch $\vec B=\mu\cdot\vec H$ (2) folgt somit auch ein Zusammenhang für die magnetische Flussdichte.

Aus der Untersuchung von Stoffmagnetismus, wie es in diesem Versuch auch an durchgeführt wurde, folgt eine
Klassifizierung verschiedener Stofftypen in Para-, Dia- und Ferromagnetismus.Da die relative Permeabilität $\mu_r$ in allen drei Typen konstant ist,
folgt zur Unterteilung eine davon abgeleitete Klassifizierung, die Magnetisierung $\vec M=\chi\cdot\vec H$ mit der magnetischen Suszeptibilität
$\chi=\mu_r -1$.

Bei Diamegneten gilt $\chi <0\; ; \;|\chi|\ll 1$. In einem angelegten B-Feld ist die Magnetisierung $\vec M$ dem B-Feld entgegengerichtet, wodurch es
eine abschwächende Wirkung hat. Diamegneten bewegen sich aus Bereichen mit hoher FEldstärke zu Bereichen mit kleinerer.
Paramagnetische Stoffe hingegen haben $\chi > 0\; ; \;|\chi|\ll 1$. Die Magnetisierung richtet sich parallel zu einem äußeren B-Feld aus,
wodurch sie eine verstärkende Wirkung hat. Paramagneten werden zu Bereichen hoher Feldstärke angezogen.
Bei Stoffen des Ferrmagnetismus gilt der Zusammenhang (2) nicht mehr, da die relative Permeabilität $\mu_r$ sehr hoch ist. Es gilt damit
$\chi > 0\; ; \;|\chi|\gg 1$. Ferromagneten richten sich, analog zu Paramagneten, parallel zum angelgten Magnetfeld aus udn verstärken dieses
stark. 
Bei Ferromagneten ist die stoffabhängige Permeabilität allerdings nicht konstant, sondern davon abhängig in welchem Zustand es bereits
einem Magnetfeld mit dessen Magnetfeldstärke asugesetzt worden war. Es ist dadurch zustandsabhängig.
Eine Hysteresekurve, die die Magnetisierung auf das äußere Magnetfeld aufträgt, beschreibt das Verhalten von Ferromagneten in
angelegten Magnetfeldern und wird im folgenden beschrieben. Die magnetischen Dipolmomente von Ferromagneten sind in einem angelegten
Magnetfeld parallel zu diesem. In einzelnen Bereichen,die man Weiß'sche Bezirke nennt, richten sich diese Dipolmomente parallel zueinander aus.
Im unmagnetisierten Zustand sind die Dipole, aufgrund der thermischen Bewegungen, statistisch verteilt.

%%%%%%%%%%%%%%%%%%%%%%%%%%%%%%%%%%%%%%%%%%

%HIER VIELLEICHT HYSTERESEKURVE EINFÜGEN

%%%%%%%%%%%%%%%%%%%%%%%%%%%%%%%%%%%%%%%%%%

Wenn ein äußeres Magnetfeld angelegt wird, in dem sich der zu untersuchende Stoff befindet, dann steigt die Magnetisierung der Probe bis
auf einen Sättigungswert an. Drosselt man das Magnetfeld nun, sinkt auch die Magnetisierung des Stoffes. Bei einem abgeschalteten Magnetfeld jedoch,
ist sie nicht null ($\vec M(\vec B = 0)\neq0$) und es bleibt eine Remanenz.
Diese kann durch bilden eines Gegenfeldes, der Koerzitivkraft, wieder aufgehoben werden. Erhöht man das Gegenfeld weiter, so bildet sich ein 
negativer Sättigungswert.
Durch erhöhen des äußeren Magnetfeldes, bildet sich eine parallele Kurve, wie auch bei der Erhöhung des Gegenfeldes. Sie erreicht den
ursprünglichen, positiven Sättigungswert. Eine Hysteresekurve hat je nach Stoff eine unterschiedliche Form.



\cite{sample}