\section{Theorie}
\label{sec:Theorie}

\subsection{Allgemeine Relaxationsgleichung}
Wenn ein System aus seinem Ausgangszustand ausgelenkt und ohne jegliche Oszillation in denselben zurückkehrt, treten Relaxationserscheinungen auf. Diese lassen sich durch
eine Differentialgleichung der Form
\begin{equation}
    \label{eqn:dA/dt}
    \frac{\symup{d}A}{\symup{d}t} = c[A(t)-A(\infty)],
\end{equation}
für die Änderungsgeschwindigkeit einer allgemeinen Größe A beschreiben.
Durch Umformung folgt daraus
\begin{equation}
    \label{eqn:AllgemeineRelaxationsgleichung}
    A(t)=A(\infty)+[A(t)-A(\infty)]e^{ct}.
\end{equation}


\subsection{Entladekurve eines Kondensators}
Die Spannung $U_C$, die aufgrund der Ladung $Q$ am Kondensator anliegt, ist durch die Kapazität $C$ bestimmt als
\begin{equation*}
    U_C=\frac{Q}{C}.
\end{equation*}
Der Strom $I$, der aufgrund der Ladungsdifferenz durch den Widerstand $R$ fließt, ist nach dem Ohm'schen Gesetz gegeben als
\begin{equation*}
    I=\frac{U_C}{R}.
\end{equation*}
Dadurch ist die zeitliche Änderung der Ladung auf dem Kondensator bestimmt als
\begin{equation*}
    \symup{d}Q=-I\symup{d}t
\end{equation*}
Aus diesen Bezügen ergibt sich die DGL
\begin{equation}
    \label{eqn:dQ/dt}
    \frac{\symup{d}Q}{\symup{d}t}=-\frac{Q(t)}{RC},
\end{equation}
die verglichen mit Gleichung (\ref{eqn:dA/dt}) eine hohe Ähnlichkeit aufweist. Da der Grenzwert $Q(\infty)$ nicht zu erreichen ist, kann er vernachlässigt werden und die Lösung
der DGL ist somit gegeben durch
\begin{equation}
    \label{eqn:Entladung}
    Q(t)=Q(0)e^{-\frac{t}{RC}}.
\end{equation}


\subsection{Relaxationsverhalten bei angelegten Wechselspannungen}
Im Allgemeinen ist Wechselspannung definiert durch die Funktion
\begin{equation*}
    U(t)=U_0 \cdot cos(\omega t).
\end{equation*}
Da eine Phasenverschiebung zwischen der eingehenden Spannung des Sinusgenerators und der verzögerten Spannung des Kondensators entsteht,
ist die ausgehende Spannung bestimmt als
\begin{equation*}
    U_{C}(t)=A(\omega)\cdot cos(\omega t +\varphi),
\end{equation*}
wobei $A$ die Kondensatorspannungsamplitude beschreibt.
Des Weiteren gilt
\begin{equation}
    \label{eqn:I(t)}
    I(t)=\frac{\symup{d}Q}{\symup{d}t}=C\frac{\symup{d}U_C}{\symup{d}t}.
\end{equation}
Durch die Kirchhoffschen Gesetzen gilt für den RC-Kreis
\begin{equation}
    \label{eqn:Kirchhoff}
    U(t)=U_{R}(t)+U_{C}(t).
\end{equation}
Aus den vorherigen Formeln (\ref{eqn:dQ/dt}), (\ref{eqn:I(t)}) und (\ref{eqn:Kirchhoff}) erhält man mit weiteren Umformungen
\begin{equation}
    \label{eqn:Amplitude}
    A(\omega)=\frac{U_0}{\sqrt{1+\omega^2 R^2 C^2}}.
\end{equation}


\subsection{Integrationsverhalten des RC-Kreises}
\label{subsec:int}
Ein RC-Kreis kann nur dann als Integrator funktionieren, wenn $\omega >> \frac{1}{RC}$ gilt.
Hierzu lässt sich Gleichung (\ref{eqn:Kirchhoff}) umschreiben zu
\begin{align*}
    U(t)&=R \cdot I(t)+U_{C}(t)\\
        &=RC\cdot \frac{\symup{d}U_{C}(t)}{dt}+U_{C}(t).
\end{align*}
Unter Berücksichtigung der Bedingung $\omega >>\frac{1}{RC}$ wird die Gleichung gelöst zu
\begin{equation}
    \label{eqn:Kondensatorspannung}
    U_{C}(t)=\frac{1}{RC}\int_{0}^{t} U(t')\symup{d}t'.
\end{equation}